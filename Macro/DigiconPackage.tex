%%%% 古いコマンド等の警告 %%%%
	\RequirePackage[l2tabu, orthodox]{nag}
%%%% 基本的な設定 %%%%
	\documentclass[a4paper]{bxjsarticle}
	\usepackage[local, libertine, yuwin, final]{DigiconPreamble}
	\usepackage{enumitem}
	% Package and Commnad's Format Macro
		\newcommand{\pkg}[1]{\textsf{#1}}
		\newcommand{\cmd}[1]{\textsf{#1}}
		\newcommand{\cmdop}[1]{$\langle$\textit{#1}$\rangle$}
	% Title
		\title{The \pkg{Digicon} Package}
		\author{Shin Kawai}
%%%% 以下のPeambleは内容によって編集すること %%%%
	% 各種コマンドの定義・再定義
	\makeatletter
	%%%% しおり、リンク作成用 %%%%
		% タイトル、著者、主題、キーワードを編集
		\usepackage{ifpdf}
		\usepackage{ifluatex}
		\usepackage{ifxetex}
		\ifpdf
			\ifluatex
			\usepackage[pdfencoding=auto]{hyperref}
			\else
			\usepackage[whole,autotilde]{bxcjkjatype}
			\usepackage[unicode]{hyperref}
			\fi
		\else
			\ifxetex
			\usepackage{hyperref}
			\else
			\usepackage[dvipdfmx]{hyperref}
			\usepackage{pxjahyper}
			\fi
		\fi
		\hypersetup{%
		bookmarksnumbered=true,%
		colorlinks=false,%
		setpagesize=false,%
		pdftitle={\@title},%
		pdfauthor={\@author},%
		pdfsubject={Digicon Package Manual},%
		pdfkeywords={Digicon; LaTeX; Macro;}}
		%
	%%%% onlyamsmathによるtikzとの競合を回避 %%%%
		\let\@@tikzpicture\tikzpicture
		\def\tikzpicture{\catcode`\$=3 \@@tikzpicture}
		\global\let\tikz@ensure@dollar@catcode=\relax
	\makeatother
	% ヘッダーとフッターの拡張用
		\usepackage{fancyhdr}
		\pagestyle{fancy}
		% ヘッダーとフッターの編集
			\lhead{\rightmark}
			\chead{}
			\rhead{\leftmark}
			\lfoot{}
			\cfoot{-\ \textsf{\thepaged}\ -}
			\rfoot{\textsf{コンパイル日時:\todayd}}
			\renewcommand{\headrulewidth}{0pt}
%%%% 本文開始 %%%%
\begin{document}

%%%% Title %%%%
	\maketitle

%%%% TOC %%%%
	\tableofcontents

%%%% DigiconPreamble.sty %%%%
	\section{DigiconPreamble.sty}
		\subsection{Introduction}
			\begin{description}[style=nextline]
				\item [Overview]
					このパッケージは...
				\item[Other Required Packages]
					このパッケージは次のパッケージを必要とする.
					\begin{itemize}
						\item \pkg{xxx}
					\end{itemize}
				\item[Set Up and Options]
					このパッケージは次のようにして読み込む.
					\begin{quote}
						\cmd{\textbackslash usepackage[\cmdop{local/sharelatex}, \cmdop{libertine/gyre}, \cmdop{yuwin/ipaex}, \cmdop{final/draft}]{DigiconPreamble}}
					\end{quote}
					各オプションの説明を次に与える.
					下線の設定がデフォルトのオプションである.
					\begin{description}[style=nextline]
						\item[\cmd{\underline{local}/sharelatex}]
					\end{description}
			\end{description}
			
		\subsection{List of Commands}\end{document}